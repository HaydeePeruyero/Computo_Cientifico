% Options for packages loaded elsewhere
\PassOptionsToPackage{unicode}{hyperref}
\PassOptionsToPackage{hyphens}{url}
%
\documentclass[
]{book}
\usepackage{amsmath,amssymb}
\usepackage{iftex}
\ifPDFTeX
  \usepackage[T1]{fontenc}
  \usepackage[utf8]{inputenc}
  \usepackage{textcomp} % provide euro and other symbols
\else % if luatex or xetex
  \usepackage{unicode-math} % this also loads fontspec
  \defaultfontfeatures{Scale=MatchLowercase}
  \defaultfontfeatures[\rmfamily]{Ligatures=TeX,Scale=1}
\fi
\usepackage{lmodern}
\ifPDFTeX\else
  % xetex/luatex font selection
\fi
% Use upquote if available, for straight quotes in verbatim environments
\IfFileExists{upquote.sty}{\usepackage{upquote}}{}
\IfFileExists{microtype.sty}{% use microtype if available
  \usepackage[]{microtype}
  \UseMicrotypeSet[protrusion]{basicmath} % disable protrusion for tt fonts
}{}
\makeatletter
\@ifundefined{KOMAClassName}{% if non-KOMA class
  \IfFileExists{parskip.sty}{%
    \usepackage{parskip}
  }{% else
    \setlength{\parindent}{0pt}
    \setlength{\parskip}{6pt plus 2pt minus 1pt}}
}{% if KOMA class
  \KOMAoptions{parskip=half}}
\makeatother
\usepackage{xcolor}
\usepackage{longtable,booktabs,array}
\usepackage{calc} % for calculating minipage widths
% Correct order of tables after \paragraph or \subparagraph
\usepackage{etoolbox}
\makeatletter
\patchcmd\longtable{\par}{\if@noskipsec\mbox{}\fi\par}{}{}
\makeatother
% Allow footnotes in longtable head/foot
\IfFileExists{footnotehyper.sty}{\usepackage{footnotehyper}}{\usepackage{footnote}}
\makesavenoteenv{longtable}
\usepackage{graphicx}
\makeatletter
\def\maxwidth{\ifdim\Gin@nat@width>\linewidth\linewidth\else\Gin@nat@width\fi}
\def\maxheight{\ifdim\Gin@nat@height>\textheight\textheight\else\Gin@nat@height\fi}
\makeatother
% Scale images if necessary, so that they will not overflow the page
% margins by default, and it is still possible to overwrite the defaults
% using explicit options in \includegraphics[width, height, ...]{}
\setkeys{Gin}{width=\maxwidth,height=\maxheight,keepaspectratio}
% Set default figure placement to htbp
\makeatletter
\def\fps@figure{htbp}
\makeatother
\setlength{\emergencystretch}{3em} % prevent overfull lines
\providecommand{\tightlist}{%
  \setlength{\itemsep}{0pt}\setlength{\parskip}{0pt}}
\setcounter{secnumdepth}{5}
\usepackage{booktabs}
\usepackage{amsthm}
\makeatletter
\def\thm@space@setup{%
  \thm@preskip=8pt plus 2pt minus 4pt
  \thm@postskip=\thm@preskip
}
\makeatother
\ifLuaTeX
  \usepackage{selnolig}  % disable illegal ligatures
\fi
\usepackage[]{natbib}
\bibliographystyle{apalike}
\IfFileExists{bookmark.sty}{\usepackage{bookmark}}{\usepackage{hyperref}}
\IfFileExists{xurl.sty}{\usepackage{xurl}}{} % add URL line breaks if available
\urlstyle{same}
\hypersetup{
  pdftitle={Temas Selectos de Análisis Numérico y Computación Científica: Computo científico para el análisis de datos},
  pdfauthor={Haydeé Peruyero},
  hidelinks,
  pdfcreator={LaTeX via pandoc}}

\title{Temas Selectos de Análisis Numérico y Computación Científica: Computo científico para el análisis de datos}
\author{Haydeé Peruyero}
\date{2024-01-24}

\begin{document}
\maketitle

{
\setcounter{tocdepth}{1}
\tableofcontents
}
\hypertarget{temas-selectos-de-anuxe1lisis-numuxe9rico-y-computaciuxf3n-cientuxedfica-computo-cientuxedfico-para-el-anuxe1lisis-de-datos}{%
\chapter{Temas Selectos de Análisis Numérico y Computación Científica: Computo científico para el análisis de datos}\label{temas-selectos-de-anuxe1lisis-numuxe9rico-y-computaciuxf3n-cientuxedfica-computo-cientuxedfico-para-el-anuxe1lisis-de-datos}}

Curso del posgrado conjunto en Ciencias Matemáticas PCCM UNAM UMICH 2024-2

\hypertarget{temario}{%
\section{Temario}\label{temario}}

\begin{enumerate}
\def\labelenumi{\arabic{enumi}.}
\item
  Git y Github
\item
  Shell
\item
  Python
\item
  SQL
\item
  Power BI
\item
  R
\item
  Estadística multivariada
\item
  Análisis de regresión
\end{enumerate}

\hypertarget{referencias}{%
\section{Referencias}\label{referencias}}

{[}1{]} Arnold, Jeremey. Learning Microsoft Power BI, O'Reilly Media, Inc.

{[}2{]} Beaulieu, Alan. Learning SQL, O'Reilly Media, Inc., 2020

{[}3{]} Bruce, Peter, Bruce, Andrew and Gedeck, Peter. Practical Statistics for Data Scientists, O'Reilly Media, Inc., 2020.

{[}4{]} Crawley, Michael J. The R book. John Wiley \& Sons, 2012.

{[}5{]} McKinney, Wes. Python for data analysis. O'Reilly Media, Inc., 2022.

{[}6{]} Nelli, Fabio. Python Data Analytics, Apress.

{[}7{]} Wade, Ryan. Advanced Analytics in Power BI with R and Python, Apress.

{[}8{]} Wickham, Hadley, and Garrett Grolemund. R for data science: import, tidy, transform, visualize, and
model data. O Reilly Media, Inc., 2016.

{[}9{]} Zamora Saiz, Alfonso, et al.~An Introduction to Data Analysis in R: Hands-on Coding, Data Mining,
Visualization and Statistics from Scratch., Springer (2020).

{[}10{]} Software Carpentry, The Unix Shell, \url{https://swcarpentry.github.io/shell-novice/}

\hypertarget{material-interesante}{%
\section{Material interesante}\label{material-interesante}}

\begin{itemize}
\tightlist
\item
  \href{https://bookdown.org/}{Bookdown}.
\item
  \href{https://swcarpentry.github.io/r-novice-gapminder/}{Software Carpentry}.
\item
  \href{https://swcarpentry.github.io/git-novice/14-supplemental-rstudio/}{Git}
\item
  \href{https://info5940.infosci.cornell.edu/setup/git/what-is-git/}{Why Git}
\item
  \href{https://bookdown.org/yihui/rmarkdown-cookbook/}{R Markdown Cookbook}
\item
  \href{http://www.sthda.com/english/wiki/data-visualization}{STHDA}
\item
  \href{https://bookdown.org/ndphillips/YaRrr/}{YaRrr! The Pirate's Guide to R}
\item
  \href{https://link.springer.com/book/10.1007/978-3-319-53019-2}{Learn ggplot2 Using Shiny App}
\item
  \href{https://link.springer.com/book/10.1007/978-0-387-98141-3}{Ggplot2: Elegant Graphics for Data Analysis}

  \begin{itemize}
  \tightlist
  \item
    \href{https://ggplot2-book.org/index.html}{Versión online}
  \end{itemize}
\item
  \href{https://www.springer.com/series/6991/books}{Use R! Colección Springer}
\item
  \href{https://link.springer.com/book/10.1007/978-0-387-75969-2}{Lattice: Multivariate Data Visualization with R}
\item
  \href{http://www.cookbook-r.com/}{R Graphics cookbook}
\end{itemize}

\hypertarget{datacamp}{%
\section{DataCamp}\label{datacamp}}

\begin{figure}
\centering
\includegraphics{img/regular.png}
\caption{DataCamp}
\end{figure}

\hypertarget{shell}{%
\chapter{Shell}\label{shell}}

\hypertarget{manipulaciuxf3n-de-archivos-y-directorios}{%
\section{Manipulación de archivos y directorios}\label{manipulaciuxf3n-de-archivos-y-directorios}}

\hypertarget{manipulaciuxf3n-de-datos}{%
\section{Manipulación de datos}\label{manipulaciuxf3n-de-datos}}

\hypertarget{tuberuxedas-y-filtros}{%
\section{Tuberías y filtros}\label{tuberuxedas-y-filtros}}

\hypertarget{ciclos}{%
\section{Ciclos}\label{ciclos}}

\hypertarget{scripts}{%
\section{Scripts}\label{scripts}}

\hypertarget{descarga-y-limpieza-de-bases-de-datos}{%
\section{Descarga y limpieza de bases de datos}\label{descarga-y-limpieza-de-bases-de-datos}}

\hypertarget{git-y-github}{%
\chapter{Git y Github}\label{git-y-github}}

\hypertarget{repositorios}{%
\section{Repositorios}\label{repositorios}}

\hypertarget{flujo-de-trabajo-en-git}{%
\section{Flujo de trabajo en Git}\label{flujo-de-trabajo-en-git}}

\hypertarget{comparando-cambios}{%
\section{Comparando cambios}\label{comparando-cambios}}

\hypertarget{crear-ramas}{%
\section{Crear Ramas}\label{crear-ramas}}

\hypertarget{actualizando-ramas}{%
\section{Actualizando ramas}\label{actualizando-ramas}}

\hypertarget{revertir-cambios}{%
\section{Revertir cambios}\label{revertir-cambios}}

\hypertarget{resolver-conflictos}{%
\section{Resolver conflictos}\label{resolver-conflictos}}

\hypertarget{python}{%
\chapter{Python}\label{python}}

\hypertarget{tipos-de-datos}{%
\section{Tipos de datos}\label{tipos-de-datos}}

\hypertarget{flujo-de-control}{%
\section{Flujo de control}\label{flujo-de-control}}

\hypertarget{visualizaciuxf3n-de-datos}{%
\section{Visualización de datos}\label{visualizaciuxf3n-de-datos}}

\hypertarget{manipulaciuxf3n-de-bases-de-datos}{%
\section{Manipulación de bases de datos}\label{manipulaciuxf3n-de-bases-de-datos}}

\hypertarget{anuxe1lisis-exploratorio-de-bases-de-datos}{%
\section{Análisis exploratorio de bases de datos}\label{anuxe1lisis-exploratorio-de-bases-de-datos}}

\hypertarget{funciones-y-scripts}{%
\section{Funciones y scripts}\label{funciones-y-scripts}}

\hypertarget{buenas-practicas}{%
\section{Buenas practicas}\label{buenas-practicas}}

\hypertarget{procesamiento-de-alto-rendimiento}{%
\section{Procesamiento de alto rendimiento}\label{procesamiento-de-alto-rendimiento}}

\hypertarget{programaciuxf3n-en-paralelo}{%
\section{Programación en paralelo}\label{programaciuxf3n-en-paralelo}}

\hypertarget{sql}{%
\chapter{SQL}\label{sql}}

\hypertarget{bases-de-datos-y-manipulaciuxf3n}{%
\section{Bases de datos y manipulación}\label{bases-de-datos-y-manipulaciuxf3n}}

\hypertarget{explorar-datos-categuxf3ricos-y-texto-no-estructurado}{%
\section{Explorar datos categóricos y texto no estructurado}\label{explorar-datos-categuxf3ricos-y-texto-no-estructurado}}

\hypertarget{comparaciuxf3n-con-los-otros-programas}{%
\section{Comparación con los otros programas}\label{comparaciuxf3n-con-los-otros-programas}}

\hypertarget{valores-faltantes}{%
\section{Valores faltantes}\label{valores-faltantes}}

\hypertarget{combinar-bases-de-datos}{%
\section{Combinar bases de datos}\label{combinar-bases-de-datos}}

\hypertarget{power-bi}{%
\chapter{Power BI}\label{power-bi}}

\hypertarget{introducciuxf3n-a-power-bi}{%
\section{Introducción a Power BI}\label{introducciuxf3n-a-power-bi}}

\hypertarget{transformando-y-visualizando-datos}{%
\section{Transformando y visualizando datos}\label{transformando-y-visualizando-datos}}

\hypertarget{manipulaciuxf3n-de-bases-de-datos-1}{%
\section{Manipulación de bases de datos}\label{manipulaciuxf3n-de-bases-de-datos-1}}

\hypertarget{anuxe1lisis-exploratorio-de-bases-de-datos-1}{%
\section{Análisis exploratorio de bases de datos}\label{anuxe1lisis-exploratorio-de-bases-de-datos-1}}

\hypertarget{variables-categuxf3ricas-y-continuas}{%
\section{Variables categóricas y continuas}\label{variables-categuxf3ricas-y-continuas}}

\hypertarget{r}{%
\chapter{R}\label{r}}

\hypertarget{tipos-de-datos-1}{%
\section{Tipos de datos}\label{tipos-de-datos-1}}

\hypertarget{manipulaciuxf3n-de-bases-de-datos-2}{%
\section{Manipulación de bases de datos}\label{manipulaciuxf3n-de-bases-de-datos-2}}

\hypertarget{anuxe1lisis-exploratorio-de-bases-de-datos-2}{%
\section{Análisis exploratorio de bases de datos}\label{anuxe1lisis-exploratorio-de-bases-de-datos-2}}

\hypertarget{reportes-con-rmarkdown}{%
\section{Reportes con RMarkdown}\label{reportes-con-rmarkdown}}

\hypertarget{puxe1ginas-web}{%
\section{Páginas web}\label{puxe1ginas-web}}

  \bibliography{book.bib,packages.bib}

\end{document}
